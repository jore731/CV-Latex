\PassOptionsToPackage{dvipsnames}{xcolor}

\documentclass[10pt,a4paper,ragged2e]{altacv}

\geometry{left=7.5cm,right=1cm,marginparwidth=6cm,marginparsep=0.5cm,top=.5cm,bottom=1.25cm}

% Change the font if you want to, depending on whether
% you're using pdflatex or xelatex/lualatex
\ifxetexorluatex
  % If using xelatex or lualatex:
  \setmainfont{Lato}
\else
  % If using pdflatex:
  \usepackage[utf8]{inputenc}
  \usepackage[T1]{fontenc}
  \usepackage[default]{lato}
\fi

% Change the colours if you want to
\definecolor{DarkBlue}{HTML}{173f5f}
\definecolor{LightBlue}{HTML}{20639B}
\definecolor{SlateGrey}{HTML}{2E2E2E}
\definecolor{LightGrey}{HTML}{666666}
\definecolor{TAGS}{HTML}{173f5f}
\colorlet{heading}{DarkBlue}
\colorlet{accent}{DarkBlue}
\colorlet{emphasis}{SlateGrey}
\colorlet{body}{LightGrey}

% Change the bullets for itemize and rating marker
% for \cvskill if you want to
\renewcommand{\itemmarker}{{\small\textbullet}}
\renewcommand{\ratingmarker}{\faCircle}

\usepackage{url}
\usepackage{hyperref}


% ++++++++++++++++++++++++++++++++++++++++++++++++++++++++++++++++++++++++++++++
% DOCUMENT
% ++++++++++++++++++++++++++++++++++++++++++++++++++++++++++++++++++++++++++++++

\begin{document}
\name{Jorge Pulido López}
\tagline{Ingeniería Electrónica Industrial y Automática || Grado en Física}

\photoR{2.8cm}{foto}
\personalinfo{
  \email{\href{mailto:jorge.pulido@hotmail.com}{jorge.pulido@hotmail.com}}
  \phone{+34-649896212}  
  \location{Madrid, España}\\
  \href{https://www.linkedin.com/in/jorgepulido731}{\linkedin {linkedin.com/in/jorgepulido731}}
  \href{https://www.github.com/jore731}{\github{github.com/jore731}}}
  \hspace*{-10cm}\makecvheader

\reversemarginpar

\cvsection[p1sidebar]{Experiencia Laboral}
\cvevent{Desarrollador Python para Airbus Defence and Space (SHM)}{Altran}{Enero 2020 -- Actualidad}{Getafe, Madrid (España)}

\cvevent{Departamento de Ingeniería y Control}{Atlas Robots}{Mayo 2018 -- Enero 2020}{Valdemoro,Madrid (España)}

\cvevent{Empleos temporales}{University of Colorado Boulder / Randstad}{Junio 2015 -- Mayo 2018}{Boulder, CO (USA) / Madrid (España)}
\begin{itemize}
  \item Supervisor de laboratorio de revelado fotográfico
  \item Mecánico de bicicletas
  \item Organización de eventos
  \item Dependiente de tienda\\
  ...
\end{itemize}
  
\cvsection{Formación}

\cvevent{Grado en Ingeniería Electrónica Industrial y Automática}{Universidad Carlos III de
Madrid}{Septiembre 2015 -- Octubre 2019}{}
Matrícula de Honor en Mecánica de Estructuras.
\\
Matrícula de Honor en Electrónica Digital.

\divider

\cvevent{Grado en Ingeniería Electrónica Industrial y Automática}{University Of Colorado At Boulder}{Agosto 2017 -- Mayo 2018}{}
Beca de movilidad internacional en programa de intercambio con la Universidad Carlos III de Madrid.

\divider

\cvevent{Grado en Física}{Universidad Nacional de Educación a Distancia (UNED)}{Septiembre 2018 -- En curso}{}

\cvsection{Certificaciones}

\cvevent{IELTS Certificate}{Cambridge University}{10 de Septiembre de 2016}{}
C1 (Advanced)

\divider

\cvevent{Certified LabVIEW Associate Developer}{National Instruments}{11 de Enero de 2019}{}
 Número de Certificación: 100-319-15117

 \clearpage
 \newgeometry{left=1cm,right=1cm,top=1.25cm,bottom=1.25cm}

\cvskillsection{Proyectos}
  \cvevent{Desarollo full stack de aplicación Python para procesamiento de datos de vuelo de aeronaves militares (Airbus Defence and Space)}{Altran}{Enero 2020 -- Actualidad}{Getafe, Madrid (España)}
    \begin{itemize}
      \item Interfaz PyQt utilizando PyQt5 multiplataforma (Linux/Windows).
      \item Implementación de librerías externas.
      \item Multithreading y multiprocessing.
      \item Comunicación con bases de datos SQL
      \item Patrón de desarrollo Modelo-Vista-Controlador modularizado.
      \item Sector aeronáutico a nivel internacional.
    \end{itemize}
    \begin{center}
      \vspace*{-0.2cm}
      \cvtag[5]{Python}
      \cvtag[5]{PyQt5}
      \cvtag[5]{SQL}
      \cvtag[5]{Front End}
      \cvtag[4]{Linux-Red Hat}
      \cvtag[3]{Data Science}
      \cvtag[3]{NumPy}
      \cvtag[3]{Pandas}
      \cvtag[2]{JSON}
      \vspace*{-0.2cm}
    \end{center}
  \divider


  \cvevent{Diseño, desarrollo y prototipado de vehículo autónomo industrial de transporte de pallets (i+D)}{Atlas Robots}{Julio 2019 -- Enero 2020}{Valdemoro, Madrid (España)}
    \begin{itemize}
      \item Arquitectura de sistemas de sensores, actuadores y comunicaciones del vehículo.
      \item Diseño eléctrico de prototipos.
      \item Gestión general del proyecto (Plazos, entregas, documentación).
      \item Programación del autómata Wago utilizado como centro de control.
      \item Arquitectura de sistemas de seguridad eléctricos cumplimentando la normativa industrial.
      \item Diseño y configuración de sistemas de comunicación internos y externos.
    \end{itemize}
    \begin{center}
      \vspace*{-0.2cm}
      \cvtag[5]{e!cockpit}
      \cvtag[5]{LADDER}
      \cvtag[5]{ST}
      \cvtag[4]{SFC}
      \cvtag[4]{CODESYS}
      \cvtag[4]{CANOpen}
      \cvtag[4]{Ethernet/IP}\\
      \cvtag[3]{Gestión de proyectos}
      \cvtag[2]{LiDAR}
      \cvtag[2]{SLAM}
      \cvtag[1]{EPLAN}
      \vspace*{-0.2cm}
    \end{center}
  \divider

  \cvevent{Programación y puesta en marcha de robot pick and place ABB con visión artifical}{Atlas Robots}{Mayo 2019 -- Noviembre 2019}{Valdemoro, Madrid (España)}
    \begin{itemize}
      \item Desarrollo de interfaz C\# con visión artificial para reconocimiento de imagen.
      \item Comunicación robot-interfaz para envío de coordenadas de cogida en tiempo real.
      \item Programación de robot ABB para el enpaquetado de bolsas (transportador de entrada) en cajas (transportador de salida).
      \item Programación del autómata Wago utilizado como centro de control de sensores y actuadores.
    \end{itemize}
    \begin{center}
      \vspace*{-0.2cm}
      \cvtag[5]{C\#}
      \cvtag[5]{OpenCV}
      \cvtag[4]{CODESYS}
      \cvtag[4]{CANopen}
      \cvtag[3]{RobotStudio} 
      \cvtag[3]{Ethernet/IP}
      \vspace*{-0.2cm}
    \end{center}
  \divider

  \cvevent{Arquitectura, programación y puesta en marcha de células industriales robotizadas a nivel nacional e internacional}{Atlas Robots}{Mayo 2018 -- Noviembre 2019}{Valdemoro, Madrid (España)}
    \begin{itemize}
      \item Gestión general de los proyectos (Plazos, entregas, documentación, pedidos y soporte técnico).
      \item Arquitectura de sistemas de sensores, actuadores y comunicaciones del vehículo.
      \item Programación de robots (KUKA, Fanuc, ABB) y autómatas (Siemens, Omron, Wago).
      \item Diseño a medida de Sistemas de Control SCADA (ProFace, Siemens, Wago)
      \item Mantenimiento eléctrico y mecánico de las instalaciones en casa de cliente.
      \item Trato con clientes de diversas industrias (Alimenticia, química, plástica, automovilística...). 
      \item Arquitectura de sistemas de seguridad eléctricos cumplimentando la normativa industrial.
      \item Diseño y configuración de sistemas de comunicación internos y externos.
    \end{itemize}
    \begin{center}
      \vspace*{-0.2cm}
      \cvtag[5]{KRL}
      \cvtag[5]{KSS 5}
      \cvtag[5]{GP PRO EX}
      \cvtag[5]{Tia Portal}
      \cvtag[4]{ProfiNet}
      \cvtag[4]{DeviceNet}\\
      \cvtag[4]{Ethernet/IP}
      \cvtag[3]{Gestión de proyectos}
      \cvtag[2]{Profibus}
      \cvtag[1]{EPLAN}
      \vspace*{-0.2cm}
    \end{center}

\end{document}
  